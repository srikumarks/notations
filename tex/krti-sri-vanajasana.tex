% XeLaTeX can use any Mac OS X font. See the setromanfont command below.
% Input to XeLaTeX is full Unicode, so Unicode characters can be typed directly into the source.

% The next lines tell TeXShop to typeset with xelatex, and to open and save the source with Unicode encoding.

%!TEX TS-program = xelatex
%!TEX encoding = UTF-8 Unicode

\documentclass[12pt]{article}
\usepackage{geometry}                % See geometry.pdf to learn the layout options. There are lots.
\geometry{a4paper}                   % ... or a4paper or a5paper or ... 
%\geometry{landscape}                % Activate for for rotated page geometry
%\usepackage[parfill]{parskip}    % Activate to begin paragraphs with an empty line rather than an indent
\usepackage{graphicx}
\usepackage{amssymb}

% Will Robertson's fontspec.sty can be used to simplify font choices.
% To experiment, open /Applications/Font Book to examine the fonts provided on Mac OS X,
% and change "Hoefler Text" to any of these choices.

\usepackage{fontspec,xltxtra,xunicode}
\usepackage[pdftex,colorlinks=true,urlcolor=cyan]{hyperref}
\defaultfontfeatures{Mapping=tex-text}
\setromanfont[Mapping=tex-text]{Times}
\setsansfont[Scale=MatchLowercase,Mapping=tex-text]{Gill Sans}
\setmonofont[Scale=MatchLowercase]{Andale Mono}
\def \deva#1{{\fontspec{DevanagariMT}#1}}
%\newcommand\tamil{\fontspec[Script=Tamil]{InaiMathi}}
\newfontfamily{\tamil}[Script=Tamil]{InaiMathi}

% info{raga}{arohana}{avarohana}{tala}
\def \info#1#2#3#4{%
	\begin{tabular}{ll}
	\textbf{R\=ag\=a}: & #1 \\
	\textbf{\=Ar\=oha\d na}: & #2 \\
	\textbf{Avar\=oha\d na}: & #3 \\\\
	\textbf{T\=a\d l\=a}: & #4
	\end{tabular}
	}

\def \speed{0.3in}
\def \f#1{\makebox[0.075in][l]{#1}}
\def \t#1{\makebox[0.15in][l]{#1}}
\def \s#1{\makebox[\speed][l]{#1}}

\def \four#1{#1\s{}}
\def \Four#1[#2]{#1#2}
\def \thr#1{#1\s{}}
\def \Sl {\s{\d{S}}}
\def \rl {\s{\d{r}}}
\def \Rl {\s{\d{R}}}
\def \gl {\s{\d{g}}}
\def \Gl {\s{\d{G}}}
\def \ml {\s{\d{m}}}
\def \Ml {\s{\d{M}}}
\def \Pl {\s{\d{P}}}
\def \dal {\s{\d{d}}}
\def \Dl {\s{\d{D}}}
\def \nl {\s{\d{n}}}
\def \Nl {\s{\d{N}}}
\def \S {\s{S}}
\def \r {\s{r}}
\def \R {\s{R}}
\def \g {\s{g}}
\def \G {\s{G}}
\def \m {\s{m}}
\def \M {\s{M}}
\def \P {\s{P}}
\def \da {\s{d}}
\def \D {\s{D}}
\def \n {\s{n}}
\def \N {\s{N}}
\def \Su {\s{\.S}}
\def \ru {\s{\.r}}
\def \Ru {\s{\.R}}
\def \gu {\s{\.g}}
\def \Gu {\s{\.G}}
\def \mu {\s{\.m}}
\def \Mu {\s{\.M}}
\def \Pu {\s{\.P}}
\def \dau {\s{\.d}}
\def \Du {\s{\.D}}
\def \nu {\s{\.n}}
\def \Nu {\s{\.N}}
\def \p {\s{,}}
\def \w {\s{}}

\def \lagu {\s{$||$}}
\def \dhru {\s{$|$}}

\def \kutti#1{\scriptsize #1}
\def \netref#1{{\scriptsize {\tt #1} (\href{#1}{go there})}}

\title{K\d rti : ``vanaj\=asana''}
\author{Subbar\=aya \'S\=astri}
%\date{}                                           % Activate to display a given date or no date

\begin{document}
\maketitle

\info{\'sr\=i}{S R$_{2}$ M$_{1}$ P N$_{2}$ \.S}{\.S N$_{2}$ P D$_{2}$ N$_{2}$ P M$_{1}$ R$_{2}$ G$_{2}$ R$_{2}$ S}{\=adi}

\vspace{0.25 in}

%P: vanajAsana vinuta nata varadAyaki shrI lalita A: kanakAdri nivAsini nanu kaTAkSincumu kAmAkSi (svara sAhityam) tAmasamikanu salupa tagAdu vinumu kAmita phalamu losaguTaku samayamu purAri ripuniki varamulu birAna nosagitivi yanucu vini nE nI padamula goliciti C1: kalAbharaNa kalAdhari sakalAgama sannuta amba kaluSArti bhaya vibhanjani ilalOpala nI valanE galadA shrI lalitAmbilkE 2: parAku salupa rAdikanu parA vinu ninnuvinA parulevvaru nannu brOcuTaku karudA karuNAkari shrIkari ninnE nera nammiti 3: sadA dalacina dAsulakunu dAsuDa shankari madanAri sati kumAruni muda sampada kAspadamai nadiga nI pada padmamu

\begin{tabular}{l}
\textbf{Pallavi:}\\
\emph{vanaj\=asana vinut\=a nata varad\=ayaki \'sr\=i lalit\'e}\\
\\
\textbf{Anupallavi:}\\
\emph{kanak\=adri niv\=asini nanu ka\d{t}\=ak\'sincumu k\=am\=ak\'si}\\
(svara s\=ahityam)\\
\emph{t\=amasamikanu salupa tag\=adu vinumu}\\
\emph{k\=amita phalamu losagu\d{t}aku samayamu}\\
\emph{pur\=ari ripuniki varamulu bir\=ana nosagitivi}\\
\emph{yanucu vini n\'e n\=i padamula goliciti}\\
\\
\textbf{Caranam:}\\
1) \emph{kal\=abhara\d{n}a kal\=adhari sakal\=agama sannuta}\\
\emph{amba kalu\'s\=arti bhaya vibhanjani ilal\=opala}\\
\emph{n\=i valan\'e galad\=a shr\=i lalit\=ambilk\'e}\\
\\
2) \emph{par\=aku salupa r\=adikanu par\=a vinu ninnuvin\=a}\\
\emph{parulevvaru nannu br\=ocu\d{t}aku karud\=a}\\
\emph{karu\d{n}\=akari shr\=ikari ninn\'e nera nammiti}\\
\\
3) \emph{sad\=a dalacina d\=asulakunu d\=asu\d{d}a}\\
\emph{\'sankari madan\=ari sati kum\=aruni muda sampada}\\
\emph{k\=aspadamai nadiga n\=i pada padmamu}\\
\end{tabular}

%\subsection
\end{document}  
