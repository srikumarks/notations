% XeLaTeX can use any Mac OS X font. See the setromanfont command below.
% Input to XeLaTeX is full Unicode, so Unicode characters can be typed directly into the source.

% The next lines tell TeXShop to typeset with xelatex, and to open and save the source with Unicode encoding.

%!TEX TS-program = xelatex
%!TEX encoding = UTF-8 Unicode

\documentclass[12pt]{article}
\usepackage{geometry}                % See geometry.pdf to learn the layout options. There are lots.
\geometry{a4paper}                   % ... or a4paper or a5paper or ... 
%\geometry{landscape}                % Activate for for rotated page geometry
%\usepackage[parfill]{parskip}    % Activate to begin paragraphs with an empty line rather than an indent
\usepackage{graphicx}
\usepackage{amssymb}

% Will Robertson's fontspec.sty can be used to simplify font choices.
% To experiment, open /Applications/Font Book to examine the fonts provided on Mac OS X,
% and change "Hoefler Text" to any of these choices.

\usepackage{fontspec,xltxtra,xunicode}
\usepackage[pdftex,colorlinks=true,urlcolor=cyan]{hyperref}
\defaultfontfeatures{Mapping=tex-text}
\setromanfont[Mapping=tex-text]{Times}
\setsansfont[Scale=MatchLowercase,Mapping=tex-text]{Gill Sans}
\setmonofont[Scale=MatchLowercase]{Andale Mono}
\def \deva#1{{\fontspec{DevanagariMT}#1}}
\def \tamil#1{{\fontspec{InaiMathi}#1}}


% info{raga}{arohana}{avarohana}{tala}
\def \info#1#2#3#4{%
	\begin{tabular}{ll}
	\textbf{R\=ag\=a}: & #1 \\
	\textbf{\=Ar\=oha\d na}: & #2 \\
	\textbf{Avar\=oha\d na}: & #3 \\\\
	\textbf{T\=a\d l\=a}: & #4
	\end{tabular}
	}

\def \speed{0.276in}
\def \f#1{\makebox[0.069in][l]{#1}}
\def \t#1{\makebox[0.138in][l]{#1}}
\def \s#1{\makebox[\speed][l]{#1}}

\def \four#1{#1\s{}}
\def \Four#1[#2]{#1#2}
\def \Sl {\s{\d{S}}}
\def \rl {\s{\d{r}}}
\def \Rl {\s{\d{R}}}
\def \gl {\s{\d{g}}}
\def \Gl {\s{\d{G}}}
\def \ml {\s{\d{m}}}
\def \Ml {\s{\d{M}}}
\def \Pl {\s{\d{P}}}
\def \dal {\s{\d{d}}}
\def \Dl {\s{\d{D}}}
\def \nl {\s{\d{n}}}
\def \Nl {\s{\d{N}}}
\def \S {\s{S}}
\def \r {\s{r}}
\def \R {\s{R}}
\def \g {\s{g}}
\def \G {\s{G}}
\def \m {\s{m}}
\def \M {\s{M}}
\def \P {\s{P}}
\def \da {\s{d}}
\def \D {\s{D}}
\def \n {\s{n}}
\def \N {\s{N}}
\def \Su {\s{\.S}}
\def \ru {\s{\.r}}
\def \Ru {\s{\.R}}
\def \gu {\s{\.g}}
\def \Gu {\s{\.G}}
\def \mu {\s{\.m}}
\def \Mu {\s{\.M}}
\def \Pu {\s{\.P}}
\def \dau {\s{\.d}}
\def \Du {\s{\.D}}
\def \nu {\s{\.n}}
\def \Nu {\s{\.N}}
\def \p {\s{,}}

\def \lagu {\s{$||$}}
\def \dhru {\s{$|$}}

\def \kutti#1{\scriptsize #1}
\def \netref#1{{\scriptsize {\tt #1} (\href{#1}{go there})}}

\title{K\d rti : ``d\=ev\=\i{} br\=ova samayamid\=e''}
\author{\'Sy\=am\=a \'S\=astri}
%\date{}                                           % Activate to display a given date or no date

\begin{document}
\maketitle

\info{cint\=ama\d ni}{S g R g M g R g P M P D n \.S}{\.S n d P M g R S}{\=adi}

\vspace{0.25 in}

%C1: lOkajananI nApai dayalEdA mAyamma nI dAsuDu gAdA shrI kAHnci vihAriNI kalyANI EkAmrEshvaruni priya bhAmayaiyunna nikEmammA entO bhAramA vinumA nA talli

%C2: rEpu mApani ceppitE nE vinanu dEvi ika tALanu nEnu I proddu dayacEyavE kRpajUDavE nI padAbjamulE madilO sadA yeHnci nIprApu kOriyunnAnammA mudamutO mA talli

%C3: shyAmakRSNuni sOdari kaumAri shaHNkari bimbAdari gauri hEmAcalajE lalitE paradEvatE kAmAkSi ninuvunA bhmilO prEmatO kApADEvareva runnArammA nA talli


\begin{tabular}{ll}
\textbf{Pallavi:} & \\
\emph{d\=ev\=\i{} br\=ova samayamid\=e} & \deva{देवी ब्रोव समयमिदे}\\
 \emph{ati v\=egam\=e vacci} & \deva{अति वेगमे वच्चि} \\
 \emph{n\=a vetalu d\=\i rcci karu\d ni\~ncav\=e} & \deva{ना वेतलु दीर्च्चि करुणिँचवे} \\
 \emph{\'sankari k\=am\=ak\d si} & \deva{शंकरि कामाक्षि} \\
 & \\
 
\textbf{Cara\d nam 1:} & \\
\emph{l\=oka janani n\=apai daya l\=ed\=a n\=ed\=a su\d du g\=ad\=a} & \deva{लोक जननी नापै दय लेदा नेदा सुडु गादा} \\
 \emph{\'sr\=\i{} k\=a\~nc\=\i{} vih\=ari\d n\=\i{} kaly\=a\d n\=\i{}} & \deva{श्री काँची विहारिणी कल्याणी} \\
 \emph{\=ek\=ambr\=e\'svarunipriya bh\=amayaiyunna-} & \deva{एकाम्ब्रेश्वरुणिप्रिय भामयैयुन्न-} \\
 \emph{n\=\i{}k\=emamm\=a ent\=o bh\=aram\=a vinum\=a n\=a talli} & \deva{नीकेमम्मा एन्तो भारमा विनुमा ना तल्लि} \\
 \emph{(d\=ev\=\i)} & \deva{(देवी)} \\
 & \\
 
\textbf{Cara\d nam 2:} & \\
\emph{\'sy\=amak\d r\d s\d nuni s\=odari kaum\=ari bimb\=adari gauri} & \deva{श्यामकृष्णुनि सोदरि कौमारि बिंबादरि गौरि} \\
\emph{h\=em\=a p\=ahi lalit\=e parad\=evat\=a} & \deva{हेमा पाहि ललिते परदेवता} \\
\emph{k\=amak\d si ninuvin\=a bh\=umil\=o pr\=emat\=o} & \deva{कामाक्षि निनुविना भूमिलो प्रेमतो} \\
\emph{k\=ap\=ad\=evar-evarunn\=aramm\=a nannu}\footnotemark[1] & \deva{कापादेवरेवरुन्नारम्मा नन्नु} \\
\emph{(d\=ev\=\i)} & \deva{(देवी)} 
\end{tabular}

\footnotetext[1]{``\emph{nannu}'' works linguistically when repeating the phrase, but doesn't seem to connect back to ``\emph{d\=ev\=\i...}''. Ref 2 says ``\emph{n\=a talli}'' instead of ``\emph{nannu}''. ``\emph{n\=a talli}'' seems to be the correct one, since the line then (I think) translates roughly to ``which god is there to protect me, my dear mother''. (Kumar) }
 
\section*{References}
\begin{enumerate}
\item Recorded performance by Shri K. V. Narayanaswamy (primary)
\item {\scriptsize \verb|http://www.medieval.org/music/world/carnatic/lyrics/TKG/devi_brova_samayamide.html| (\href{http://www.medieval.org/music/world/carnatic/lyrics/TKG/devi_brova_samayamide.html}{go there})}
\item \href{http://en.wikipedia.org/wiki/List_of_Janya_Ragas}{List of Janya Ragas} on Wikipedia\\
{\scriptsize \verb|http://en.wikipedia.org/w/index.php?title=List_of_Janya_Ragas&oldid=342064990|  (\href{http://en.wikipedia.org/w/index.php?title=List\_of\_Janya_Ragas&oldid=342064990}{go there})}
\item Online Telugu-English dictionary\\ \netref{http://dsal.uchicago.edu/dictionaries/gwynn}
\end{enumerate}

\section*{Lyrics as given in ref 2}
\begin{tabular}{ll}
Pallavi: & \emph{d\=ev\=\i{} br\=ova samayamid\=e}\\
 & \emph{ati v\=egam\=e vacci}\\
 & \emph{n\=a vetalu d\=\i rcci karu\d ni\~ncav\=e} \\
 & \emph{\'sankari k\=am\=ak\d si}\\

Cara\d nam 1: & \emph{l\=oka janan\=\i{} n\=apai dayal\=ed\=a \textbf{m\=ayamma} n\=ed\=a su\d du g\=ad\=a} \\
 & \emph{\'sr\=\i{} k\=a\~nc\=\i{} vih\=ari\d n\=\i{} kaly\=a\d n\=\i{}} \\
 & \emph{\=ek\=ambr\=e\'svarunipriya bh\=amayaiyunna-}\\
 & \emph{n\=\i{}k\=emamm\=a ent\=o bh\=aram\=a vinum\=a n\=a talli} \\ 
 & \emph{(d\=ev\=\i)}\\
 
\textbf{Cara\d nam 2}: & \textbf{\emph{r\=epu m\=apani ceppit\=e n\=e vinanu}} \\
 & \textbf{\emph{d\=evi ika t\=a\d lanu n\=enu}} \\
 & \textbf{\emph{\=\i{} proddu dayac\=eyav\=e k\d rpaj\=u\d dav\=e}} \\
 & \textbf{\emph{n\=\i{} pad\=ambujamul\=e madil\=o sad\=a ye\~nci}} \\
 & \textbf{\emph{n\=\i pr\=apu k\=oriyunn\=anamm\=a mudamut\=o m\=a talli}} \\
 & \emph{(d\=ev\=\i)}\\

Cara\d nam \textbf{3}: & \emph{\'sy\=amak\d r\d s\d nuni s\=odari kaum\=ari \textbf{\'sankari} bimb\=adari gauri} \\ 
 & \emph{\textbf{h\=em\=acalaj\=e} lalit\=e \textbf{parad\=evat\=e}} \\ 
 & \emph{k\=amak\d s\=\i{} ninuvin\=a bh\=umil\=o pr\=emat\=o} \\ 
 & \emph{k\=ap\=ad\=evar-evarunn\=aramm\=a \textbf{n\=a talli}}\\
 & \emph{(d\=ev\=\i)} 

\end{tabular}

\end{document}  